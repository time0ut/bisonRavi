\backmatter
\chapter{Épilogue}
\epigraph{À mercredi prochain !}
{\emph{Cinématographe}, \BV}
\vfill
\pagebreak

\lettrine{I}l faut bien une fin\footnote{Le loup est d'ailleurs réputé pour la sienne.},
n'est-ce pas ?

Maintenant que vous aussi, vous
savez combien \BV\ était génial, allez (ou retournez !) lire ses
\oe{}uvres et écouter ses chansons ! Et plus vite que ça !
Et aussi, n'oubliez pas de les offrir à vos proches --- et même
au moins proches ---, répandez la bonne parole 'Pataphysique !


Avant que vous ne fermiez --- avec la satisfaction du devoir accompli --- ce document,
sachez que j'ai pris grand plaisir à le réaliser. Il n'est peut-être pas parfait,
mais j'ai atteint mon objectif: en apprendre plus sur \BV, et lire ses textes
au passage ! Je me suis également bien amusé, ce qui est un pré-requis pour
quiconque envisage de s'intéresser à \BV.

Je tiens à vous remercier de m'avoir lu jusqu'au bout, sans vous endormir
\footnote{Et si vous vous êtes endormi, eh bien tant pis.}. Vous avez été
un lecteur de premier choix\footnote{Ou au moins ayant fait preuve d'assez de
bonne volonté pour lire ceci}. Merci, donc.

Mais également:
\begin{itemize}
\item mon tuteur, M. Lionel \bsc{Brunie}. Merci.
\item les gens ayant documenté la vie et l'\oe{}vre de \BV. Merci.
\item \BV. Merci
\end{itemize}

Vous pouvez maintenant reprendre une activité normale\footnote{Je vous laisse
cependant le soin de définir le terme.}.

\vfill
\hfill 'Pataphysiquement vôtre,
\vskip 2cm
\hfill Raphaël


\hfill Córdoba, le \nb{22} gidouille \nb{139}
\vskip 2cm

\tableofcontents

