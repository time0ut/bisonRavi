% Partie 3: l'héritage
\chapter{La maladie, la mort, l'héritage}

%\section{Un héritage riche et une influence encore vive aujourd'hui}
\section{Un héritage riche}

En ayant à l'esprit toutes (ou ne serait-ce même qu'une partie) des activités,
tous les métiers qu'il a éxercé, il aurait été bien étonnant que \BV\ ne
laisse pas une trace, ne soit pas une source d'influences pour les générations
futures.
C'est effectivement le cas. Son héritage est riche et multiple, et je vais
dévelloper ici les trois principaux aspects (il faut bien choisir) qui me semblent
les plus marquants de ces onfluences.

\subsection{Culture et social}

\BV\ a laissé sa marque dans le paysage culturel et social français. Déjà
de son vivant, il marquait les esprits, étant un personnage un peu hors-norme; et
certains de ces traits, en plus de ses oeuvres, sont passés à la postérité.

\paragraph{Le Prince de Saint-Germain}

Qui n'a jamais entendu parler de Saint-Germain-des-Prés ? Je parle bien-sûr
du Saint-Germain de l'après-guerre, le lieu de rencontre des intellectuels et
des artistes parisiens: Sartre, Queneau, Prévert % TODO: citer plus
et bien d'autres. Le soir, la jeuness du tout-Paris se retrouve dans les caves
des établissements du quartier, dansant (et buvant) toute la nuit au son jazz
noir-américain. Swing, zfpok et cuite garantis sur facture !

Le surnom de «Prince de Saint-Germain» donné à \BV\ atteste de son importance
dans ce petit monde, connaissant tous (et toutes \ldots), animant avec ses amis et
ses frères les soirées endiablées, d'abord au \emph{Tabou}, puis une fois la
frénésie des premières années passées, dans l'ambiance plus feutrée du \emph{Club
Saint-Germain}.

Sa connaissance intime de Saint-Germain et de sa faune pousse un éditeur, au moment
ou Saint-Germain et les bachannales qui s'y déroulent deviennent plus connues du
grand public, de demander au «Prince» un «Guide de Saint-Germain-des-Prés».
L'ouvrage, prévu avec force descriptions farfelues et illustrations des gens et
lieux, ne fut hélas pas publié, l'éditeur ayant fait faillite entre-temps.

C'est également dans ces clubs que \BV\ acceuille ses idole du jazz que sont
Miles \bsc{Davis}, Duke \bsc{Ellington} (son dieu), et bien d'autres \ldots

\paragraph{Langage}

Amateur de langage et de jeux de mots, expérimentateur du verbe et néologiste
patenté, écrivain et homme public: il n'est pas surprenant que des exprseeions
de son cru nou parviennent.
Le meilleur exemple est sans aucun doute l'utilisation du mot «tube».

C'est lors d'une réunion de travail chez Philips en 1957, alors qu'il y ait
directeur artistique, qu'il propose ce mot pour désigner un succès populaire,
ou une chanson qui est assurée d'avoir du succès, parfois malgré l'ineptie du
texte ou la qualité musicale. Boris proposait ce mot pour remplacer l'alors
usité «saucisson». Devant la supériorité objective du candidat, il n'est pas
surprenant qu'il est été adopté --- difficile d'imaginer un \emph{disc jokey}
annoncer le dernier «saucisson» de l'été ! Par la même occasion, \BV\ a
fourni une alternative viable au \emph{hit} anglais. Cocorico.

\paragraph{La génération 68}

La première large reconnaissance littérarire de \BV\ --- des oeuvres signées
de son vrai nom s'entend --- fût apportée par la jeunesse de la fin des années 60.
Se sentant représentés par cet auteur si anticonformiste, anticonventionel, dont
le destin tragique à gonflé le myhte de rêveur à le jeunesse éternelle, \BV\ 
et son oeuvre --- en particulier \emph{L'écume de jours} --- ont influencé toute une
génération. En avance sur son temps comme souvent --- même lorsqu'il s'agit de
mourrir ! --- \BV\ n'a malheureusement pas connu cette gloire méritée.

\subsection{Musique}

\subsection{Théâtre}


