% Partie 1: présentation du personnage

\section{Boris Vian, Bison Ravi, et tout leurs amis}

% Intro

Il est difficile de parler de Boris Vian. Peut-être parce qu'il
est difficile de lui coller une seule étiquette. Un seul nom,
même. «Boris Vian» pour l'état civil, «Bison Ravi» -- anagramme
de «Boris Vian» pour les proches, «Vernom Sullivan» pour certains
livres, et des dizaines d'autres pseudonymes en tant que chroniqueur.
% TODO: lien liste des pseudonymes (annexes ?)
Mais pour évoquer le personnage, on peut déjà s'intéresser à l'Histoire,
et à son histoire.
% TODO: Annexe frise chrono ?

\subsection{Contexte historique}
Il est nécessaire de décrire le contexte historique de la vie
de Boris Viqn pour comprendre certaines forces externes qui
ont pu avoir une influence significative sur sa vie, ainsi que
sur son oeuvre.

Boris Vian est né peu après la seconde guère mondiale. En 1929, c'est
la crise avec le crash de Wall Street. En 1940, l'Allemagne envahit la
partie nord de la France. Toute une partie de la culture est alors
interdite et censurée, notament le jazz, d'origine noire-américaine.

\subsection{Famille et éducation}

Boris Vian est issu d'une famille riche. Son père, Paul Vian, est rentier
depuis ses 20 ans. Sa mère est l'héritière d'une riche famille de l'industrie
du papier.

Les Vian habitent une grande maison, «Les Fauvettes», à Ville d'Avray, dans la
banlieue de Paris, près du parc de Saint-Cloud. Le plus important est le loisir,
le divertissement, tout ce qui est agréable. Les enfants Vian vivent ainsi coupés
du monde extérieur: la politique, la religion, ou tout autre sujet sérieux n'entre
pas dans ce petit monde clos. On profite de la vie.

Cette maison n'est pas le seul paradis des Vian. Tous les étés, ils se rendent
à Landemer, dans le Cotentin, où les enfants peuvent jouer tout l'été sur la
plage privée de la propriété apportée par la Mme. Vian. 

\subsection{Vie publique, vie privée}

