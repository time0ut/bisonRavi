\chapter{Images}
\chapter{Dossier}
\section{Introduction}
\section{Boris Vian}
\subsection{Introduction}
\subsection{sous-partie 1}
\subsection{sous-partie 2}
\subsection{sous-partie 3}
\subsection{Conclusion}
\section{Bison Ravi}
\subsection{Introduction}
\subsection{sous-partie 1}
\subsection{sous-partie 2}
\subsection{sous-partie 3}
\subsection{Conclusion}
\section{Partie 3}
\subsection{Introduction}
\subsection{sous-partie 1}
\subsection{sous-partie 2}
\subsection{sous-partie 3}
\subsection{Conclusion}
\section{Conclusion}
\chapter{Citations}
\section{Queneau}

«Boris fût toujours futur, sa mort c'est du passé.»



\chapter{Bibliographie}

Livres:
	. «Les vies parallèles de Boris Vian», Noël Arnaud (5ème édition)
	. Livres de Boris Vian

Vidéo:
	. «La vie jazz», documentaire de Philippe Coly

Audio:
	. «Boris Vian», 2000 ans d'Histoire
	. «Si c'était à recommencer» (30 juin 1956), émission présentée par Robert Bogdali, avec Boris Vian
	. Chansons de Boris Vian

Multimédia:
	. Différents sites Internet: Arte, Wikipedia



\section{Internet}
\subsection{Wikipédia}
\subsubsection{Bibliographie}
\subsection{Nouvel Obs}
\subsubsection{Interview Michelle}
\subsection{Arté}
\section{Audio}
\subsection{Interview}
\subsection{2000 ans d'histoire}
\section{Livre}
\subsection{Les vies parallèles de Boris Vian}

«Les vies parallèles de Boris Vian», Noël Arnaud (5ème édition)



\section{Vidéo}
\subsection{La vie Jazz}

«La vie jazz», documentaire de Philippe Coly



\chapter{Boris Vian}
\section{Pseudonymes}
\section{Œuvres}
